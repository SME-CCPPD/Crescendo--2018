\documentclass{scrartcl}
\usepackage{xspace}
\usepackage[italian]{babel}
\usepackage{gitinfo2}

\newcommand{\reltag}{v.\gitAbbrevHash}

\title%
{%
  Quale Musica Elettronica?\\%
  {\tiny (\reltag)}
}

\author%
{%
  Nicola Bernardini\\
  Conservatorio ``C.Pollini'' Padova%
}%

\begin{document}

\maketitle

\begin{abstract}
  Questa escursione parte da un dato di fatto evidente e triviale: oggi
  l'informatica pervade ogni aspetto della vita umana -- dagli aspetti pi\`u
  eclatantemente pubblici sino a quelli pi\`u intimamente privati, passando
  attraverso tutto ci\`o che ne popola il mezzo. Questa espansione ha
  indubbiamente numerosi aspetti positivi e non pu\`o certo essere
  stigmatizzata -- d'altra parte molto dell'interesse che riveste la
  disciplina ``Musica Elettronica'' \`e senza dubbio legato alla
  mercificazione pervasiva dell'informatica.

  Tuttavia, da tempo si \`e profilato un problema di importanza sostanziale
  nell'insegnamento della musica elettronica: in un'era nella quale
  \emph{qualsiasi elemento musicale} (esecuzione, tipologia, editoria,
  comunicazione, ecc.) passa attraverso l'elettronica -- di cosa si parla
  quando si parla, appunto, di \emph{musica elettronica}?

  Questo intervento intende presentare una breve analisi del problema
  proponendone una soluzione tutta da discutere.
\end{abstract}

\end{document}

% Oggi qualsiasi sistema di produzione sonora passa da tecnologie elettroniche. 
% Quindi tutti si sentono autorizzati ed ``esperti'' a insegnare. 
% Una cosa sono le tecnologie utilizzate, altra cosa sono i concetti - fare
% paralleli col mondo strumentale (vigili urbani che insegnano il flauto)7.
% D'altra parte
% Cosa vale la pena d'insegnare?
% Idealmente, gli studenti devono essere meglio dei propri docenti.
