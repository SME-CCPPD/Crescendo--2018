%
% Copyright (C) 2018 Nicola Bernardini nicb@sme-ccppd.org
% 
% This work is licensed under a Creative Commons License, and specifically the
% 
%   Creative Commons Attribution-ShareAlike 2.5 License
%   http://creativecommons.org/licenses/by-sa/2.5/legalcode
% 
% Check http://www.creativecommons.org/ for more information on
% Creative Commons Licenses and the Creative Commons Project.
%
% Title: ``Quale musica elettronica?''`
% 
% Abstract (italiano)
% 
% Questa escursione parte da un dato di fatto evidente e triviale: oggi
% l'informatica pervade ogni aspetto della vita umana -- dagli aspetti pi\`u
% eclatantemente pubblici sino a quelli pi\`u intimamente privati, passando
% attraverso tutto ci\`o che ne popola il mezzo. Questa espansione ha
% indubbiamente numerosi aspetti positivi e non pu\`o certo essere
% stigmatizzata -- d'altra parte molto dell'interesse che riveste la
% disciplina ``Musica Elettronica'' \`e senza dubbio legato alla
% mercificazione pervasiva dell'informatica.
%
% Tuttavia, da tempo si \`e profilato un problema di importanza sostanziale
% nell'insegnamento della musica elettronica: in un'era nella quale
% \emph{qualsiasi elemento musicale} (esecuzione, tipologia, editoria,
% comunicazione, ecc.) passa attraverso l'elettronica -- di cosa si parla
% quando si parla, appunto, di \emph{musica elettronica}?
%
% Questo intervento intende presentare una breve analisi del problema
% proponendone una soluzione tutta da discutere.
%
% Altre annotazioni
%
% Oggi qualsiasi sistema di produzione sonora passa da tecnologie elettroniche. 
% Quindi tutti si sentono autorizzati ed ``esperti'' a insegnare. 
% Una cosa sono le tecnologie utilizzate, altra cosa sono i concetti - fare
% paralleli col mondo strumentale (vigili urbani che insegnano il flauto)7.
% D'altra parte
% Cosa vale la pena d'insegnare?
% Idealmente, gli studenti devono essere meglio dei propri docenti.
%
% Set the macros below to whatever is appropriate in a given context
%

\newcommand{\imagedir}{../images}
\newcommand{\templatedir}{../../templates/beamer/sme-ccppd}
\documentclass[compress,\mode]{beamer}

\usepackage{beamerthemeSME-CCPPD}
\usepackage{beamercolorthemeSME-CCPPD}
\usepackage{beamerinnerthemeSME-CCPPD}

\usepackage{colortbl}
\usepackage[italian]{babel}
\usepackage{pgf}
\usepackage{xspace}

\usepackage{multimedia}
\usepackage{xmpmulti}
\usepackage{hyperref}
\usepackage{gitinfo2}
\newcommand{\rcstag}{rev.\gitAbbrevHash\ \ \gitAuthorDate\xspace}
\usepackage{gensymb}

\newcommand{\cpyear}{2018}
\newcommand{\cpholder}{Nicola Bernardini}
\newcommand{\cpholderemail}{nicola.bernardini@conservatoriopollini.it}

% Use some nice templates

%\beamertemplateshadingbackground{red!10}{structure!10}
\beamertemplatetransparentcovereddynamic
\beamertemplateballitem
\beamertemplatenumberedballsectiontoc

% My colors
\definecolor{notdone}{gray}{0.35}

%\usecolortheme[named=MyColor]{structure}
%\usecolortheme[named=MyColor]{structure}
\beamertemplateshadingbackground{white!10}{white!10}

\input{\templatedir/macros}

\title[Quale ME?]
{%
  Prima, durante, e dopo:\\
  Quale Musica Elettronica?\\
  {\tiny (\rcstag)}
}

\author{%
  Nicola Bernardini\\
    \href{mailto:\cpholderemail}{\cpholderemail}
}
\institute[SMERM]%
{%
  \href{http://www.conservatoriopollini.it}
     {Conservatorio di Musica ``Cesare Pollini'' -- Padova}
}
\date[Crescendo! Sassari, 05-07/04/2018]{Sassari, Convegno ``Crescendo!'' -- 05-07/04/2018}

\begin{document}
\newcounter{ms}
\setcounter{ms}{0}
  
\begin{frame}
  \titlepage
\end{frame}

\section{Introduzione}

\begin{frame}

  \begin{itemize}[<+- | alert@+>]

     \item Quale Musica Elettronica?

     \item Cosa \emph{vale la pena} insegnare?

     \item Quali strumenti?

     \item Quali studenti?

  \end{itemize}

\end{frame}

\section{Quale Musica Elettronica?}

\begin{frame}

% \framezoom<0pt,4cm><\textwidth,\textheight>(0pt,4cm)(\textwidth,6cm)
\pause
\begin{figure}
  \pgfimage[height=0.95\textheight]<+->{\imagedir/volantino_corso_dj_2018-04-05_1}
\end{figure}

\end{frame}

\begin{frame}

\begin{figure}
  \pgfimage[width=0.95\textwidth]{\imagedir/volantino_big}
\end{figure}

\end{frame}

\begin{frame}
  \frametitle<+->{Ladri di parole?}

  %
  % citare Jean-Luc Mélanchon
  %

  \uncover<+- | alert@+>{\Huge\centering Ladri di Parole!}

  \uncover<+- | alert@+>{\normalsize%
    \vspace{\baselineskip}
    ``Ci rubano le parole per impedirci di pensare''\\
    (Jean-Luc M\'elenchon in dialogo con Chantal Mouffe,
    \url{https://www.youtube.com/watch?v=FtriFMxsOWw})%
  }

\end{frame}

\begin{frame}

  \frametitle<+->{``Generi'' o ``Funzioni''?}

  \begin{itemize}[<+- | alert@+>]

     \item Generi?

     \item Funzioni: \uncover<+- | alert@+>{rito,}\uncover<+- | alert@+>{ intrattenimento,}\uncover<+- | alert@+>{ \emph{speculazione}}

  \end{itemize}

\end{frame}

\begin{frame}

  \frametitle<+->{Quale Musica Elettronica?}

  \begin{itemize}[<+- | alert@+>]

    \item l'innesto di nuove tecnologie nella musica si \`e
            sempre legato alla sperimentazione (quindi alla funzione
            speculativa e di ricerca della musica)

    \item le tecnologie elettroniche e informatiche non fanno eccezione

    \item di converso, l'intrattenimento \emph{usa} le tecnologie come
            surrogato funzionale di elementi pre--esistenti

  \end{itemize}

\end{frame}

\section{Cosa vale la pena insegnare?}

\begin{frame}

  \frametitle<+->{Cosa vale la pena insegnare?}

  \begin{itemize}[<+- | alert@+>]

     \item Regola empirica: \uncover<+- | alert@+>{si insegna tutto ci\`o la
             cui spiegazione \uncover<+->{(non \emph{l'esercizio}!)} impegna
             meno tempo dell'auto--apprendimento}
             \uncover<+- | alert@+>{per gli amici: ``non s'insegna ci\`o che
             si pu\`o pi\`u facilmente imparare da soli''}
            
  \end{itemize}

\end{frame}

\begin{frame}

  \frametitle<+->{Quindi:}

    \begin{itemize}[<+- | alert@+>]

       \item \emph{non} s'insegna a vincere alla lotteria

       \item \emph{non} s'insegna ad andare in bicicletta

       \item \emph{non} s'insegna a scrivere sui muri dei bagni pubblici

       \item \emph{non} s'insegna a fare l'amore

       \item \emph{non} s'insegna a leggere la documentazione del software

       \item \ldots

    \end{itemize}

\end{frame}

\begin{frame}

  \frametitle<+->{Invece:}

    \begin{itemize}[<+- | alert@+>]

       \item s'insegnano le radici storiche della musica elettronica

       \item le modalit\`a di pensiero compositivo sorte con l'introduzione
               delle tecnologie elettroniche e informatiche

       \item i fondamenti teorici delle tecniche di sintesi e di analisi

       \item la teoria dei segnali musicali

       \item la psicoacustica sperimentale

       \item \ldots

    \end{itemize}

\end{frame}

\begin{frame}

  \frametitle<+->{E:}

    \begin{itemize}[<+- | alert@+>]

       \item alcuni argomenti somigliano pericolosamente a materie
             dei corsi conservatoriali pi\`u tradizionalisti:

       \begin{itemize}[<+- | alert@+>]

         \item programmazione\uncover<+- | alert@+>{ $\Rightarrow$ contrappunto, mottetto e fuga}

         \item elettroacustica\uncover<+- | alert@+>{ $\Rightarrow$ armonia}

         \item acustica e (soprattutto) psicoacustica\uncover<+- | alert@+>{ $\Rightarrow$ orchestrazione e strumentazione}

         \item \ldots

       \end{itemize}

    \end{itemize}

\end{frame}

\section{Quali strumenti?}

\refstepcounter{ms}
\begin{frame}

   \frametitle<+->{La scelta degli strumenti (\arabic{ms})}

   \begin{itemize}[<+- | alert@+>]

     \item gli strumenti condizionano il pensiero

    \item una scelta consapevole \`e quindi innanzitutto una scelta
            politico--ideologica:

      \begin{itemize}[<+- | alert@+>]

        \item \emph{non} si usano strumenti non--liberi e chiusi\ldots

        \item \ldots ma s'insegnano le logiche di sviluppo libero, aperto e condiviso

        \item \emph{non} si predilige un solo strumento con il quale si fa
                tutto\ldots

        \item \ldots ma s'insegna a scegliere lo strumento adatto per ciascun
                problema (con il corollario dell'evoluzione strumentale)

        \item \ldots

      \end{itemize}

   \end{itemize}

\end{frame}

\refstepcounter{ms}
\begin{frame}

   \frametitle<+->{La scelta degli strumenti (\arabic{ms})}

   \begin{itemize}[<+- | alert@+>]

     \item \emph{non} si spinge gli studenti a dotarsi di piattaforme
             informatiche costose e strutturalmente inadatte al lavoro
             professionale\ldots

     \item ma s'insegna loro a utilizzare qualsiasi piattaforma (?) in maniera
             seria e professionale

   \end{itemize}

\end{frame}

\setcounter{ms}{0}
\refstepcounter{ms}

\begin{frame}

  \frametitle<+->{Quali studenti?}

  \begin{itemize}[<+- | alert@+>]

    \item Va operata una scelta: \uncover<+->{corsi
            esclusivi\ldots}\uncover<+->{ o corsi inclusivi?}

    \item Esiste ancora la nozione di ``talento musicale''?

    \item \ldots o la qualit\`a si trova all'interno della quantit\`a?
            \uncover<+->{(nozione pi\`u faticosa da gestire, per\`o)}

  \end{itemize}

\end{frame}

\refstepcounter{ms}
\begin{frame}

  \frametitle<+->{Quali sono i migliori?}

  \begin{center}
    Gli studenti migliori sono quelli che si diplomano
    essendo diventati pi\`u bravi dei propri docenti.
  \end{center}

\end{frame}

\end{document}
